\documentclass[10pt,a4wide]{article}
\usepackage{geometry}
\geometry{a4paper}
\usepackage{xcolor}
\usepackage{url}
\usepackage{graphicx}
\usepackage{hyperref}
\DeclareGraphicsRule{.tif}{png}{.png}{`convert #1 `dirname #1`/`basename #1 .tif`.png}

\def\warn#1{\textcolor{red}{#1}} % macro used in the notes in this file
\input pow-reed-xrefs.aux % cross-references generated by the reed tool for use in the notes here
\makeatletter % this file uses cross-references in the REED paper, REED.tex ...
	\input ../../REED-paper/REED.aux
\makeatother
\title{Princess of Wales Hospital blood glucometer case\\Version v3}\author{Harold Thimbleby}
\date{10 September 2025}
\begin{document} \maketitle
\def\colorflag#1{\fboxrule=0.3pt
\fboxsep=0pt
\hbox{\vrule height 2.2ex width 1pt
\raise 1ex\hbox{\fbox{\color{#1}\hbox{\vrule width .5ex height 1ex depth 0ex}}}%
\hskip -.6pt\raise .9ex\hbox{\fbox{\color{#1}\hbox{\vrule width .5ex height 1ex depth 0ex}}}%
\hskip -.6pt\raise 1ex\hbox{\fbox{\color{#1}\hbox{\vrule width .7ex height 1ex depth 0ex}}}}%
}\setlength{\fboxsep}{1.5pt}

\begin{abstract}
\noindent\textcolor{blue}{\sf\textbf{Keyphrases:}
Audit;
Backups;
Corruption of evidence;
CSV;
Cybersecurity;
Error logs;
PrecisionWeb database;
Testing.}
\vskip .5ex
\noindent
Automatically-generated evidence narrative file revisiting the 2014 Princess of Wales Hospital blood glucometer case. For an overview of the background to the case, with a full description of the meaning of the diagram and further details of its associated narrative, go to the REED web site home page at 
\textcolor{blue}{\url{https://harold.thimbleby.net/reeds}}\\ \\

\setbox0=\hbox{\textcolor{blue}{\url{https://harold.thimbleby.net/reeds/REED.pdf}}}
\begin{figure}\includegraphics[width=\textwidth]{../../REED-paper/figures/fig6-reedv3.pdf}\\
{\sf REED v3 diagram (there is only one new node added in v3). For convenience, this is a copy of figure \ref{fig:v3} of the REED paper itself, \copy0\ (page \pageref{fig:v3})}
\label{thefigure}
\end{figure}\newpage

\end{abstract}


\section*{Quick overview}
{\large\sf\noindent \begin{tabular}{@{}|rlcrl|}\hline 
\multicolumn{5}{@{}|l|}{Combined versions: v2; v3
)}\\\hline
 28 & nodes & --- & 23 & highlighted\\&&& 28 & with notes\\

    31 & arrows & --- & 7 & with notes\\\hline
\end{tabular}
\vskip 4mm
\noindent
\begin{tabular}{|ll|}\hline 
REED file&Digital signature\\ \hline 
lib/pow-reed & 38fc75953b3876525d37d1f3a96e0f86 \\
\hline
\end{tabular}

\vskip 2mm\noindent
\emph{\raggedright\small If a REED file has been changed since this report was made, then its digital signature will now be different. You can run the REED tool with flag `--s' on any file to confirm its signature}.

}


\vskip 5mm\noindent
There are 2 weakly connected components (i.e., there are 2 independent REED diagrams not connected to each other with any arrows, regardless of the directions of the arrows).

\begin{description}
\item[$\rightarrow$] \hyperlink{component1-narrative}{Narrative evidence for component 1}
\item[$\rightarrow$] \hyperlink{component2-narrative}{Narrative evidence for component 2}
\end{description}

\section*{23 highlighted nodes}
\setbox0=\hbox{\colorflag{white}}%
\dimen0 = \ht0 \advance \dimen0 by 1ex \ht0 = \dimen0
% to fix for textwidth...
\newdimen \colwid \colwid = \textwidth \advance \colwid by -1.2in
\newdimen \headwid \headwid = \textwidth \advance \headwid by -1in
\begin{tabular}{@{}|cp{1.45cm}p{\colwid}|}
\hline
\multicolumn{3}{@{}|p{\headwid}|}{\bf Highlighting key for Princess of Wales Hospital blood glucometer case} \\ \hline
\vphantom{\copy0}\colorflag{blue}&\hbox{blue} \hbox{used 11 times}&Version 3 uses blue to highlight the impact of critical new evidence introduced by the Abbott engineer. Only one node was explicitly highlighted blue, but the REED tool is used to automatically cascade the blue highlight to all affected evidence.\\&&\scriptsize (blue used explicitly once before cascading)\\ 
\colorflag{red}&\hbox{red} \hbox{used 2 times}&Specific, relevant computer problems as already admitted in evidence. Use of non-forensic tools like Excel (which, for instance, allows rows of data to be deleted \emph{without leaving any record of changing the data\/}) used to process the evidence. In short, any evidence highlighted in red is unreliable.\\&&\scriptsize (red used explicitly 12 times before cascading)\\ 
\colorflag{white}&\hbox{white} \hbox{used 6 times}&No evidence is available yet. This may or may not be considered a problem after further information is provided.\\&&\scriptsize (white used explicitly 16 times before cascading)\\ 
\colorflag{yellow}&\hbox{yellow} \hbox{used 4 times}&Evidential problems that have not yet been resolved. Concerns need to be addressed in cross-examination.\\&&\scriptsize (yellow used explicitly 16 times before cascading)\\ 
\hline \end{tabular}
\vskip 4ex
\noindent\begin{tabular}{@{}llll}
\colorflag{blue}&Cascaded node&v3-2.2&\hyperlink{Abbott_engineer}{Unsupervised Abbott engineer}\\
\colorflag{blue}&\scriptsize (yellow before cascade)&v2-3.3&\hyperlink{unknownEdits}{Unknown manual edits to PrecisionWeb database}\\
\colorflag{blue}&\scriptsize (red before cascade)&v2-5.2&\hyperlink{7}{Abbott PrecisionWeb database}\\
\colorflag{blue}&\scriptsize (red before cascade)&v2-6.1&\hyperlink{10}{Frequent crashes documented}\\
\colorflag{blue}&\scriptsize (yellow before cascade)&v2-6.2&\hyperlink{8}{Lots of unknown complex middleware}\\
\colorflag{blue}&\scriptsize (red before cascade)&v2-6.3&\hyperlink{11}{Police forensic database system}\\
\colorflag{blue}&\scriptsize (yellow before cascade)&v2-7.2&\hyperlink{9}{Main hospital databases}\\
\colorflag{blue}&\scriptsize (red before cascade)&v2-7.3&\hyperlink{12}{Main police evidence}\\
\colorflag{blue}&\scriptsize (yellow before cascade)&v2-8.2&\hyperlink{backups}{No known backups that could have provided evidence}\\
\colorflag{blue}&\scriptsize (white before cascade)&v2-8.3&\hyperlink{unknowndatabaselogs}{No evidence on any system reliability}\\
\colorflag{blue}&\scriptsize (white before cascade)&v2-8.4&\hyperlink{3}{Nurse-written paper records (never disclosed)}\\

\end{tabular}
\hypertarget{component1-narrative}{\section*{Node narrative evidence for component 1}}
\begin{description}
\item[$\rightarrow$] \hyperlink{component2-narrative}{All narrative for component 2}
\end{description}


\hypertarget{2}{\subsection*{ \colorflag{white} Node v2-1.2 Patients}}
 The patients on Ward 2 had limited capacity.
\vskip .5ex\vbox{\small \hskip 2em\\$\rightarrow$ \hyperlink{4}{\colorflag{white}~v2-2.4 XceedPros on Ward 2}}

\hypertarget{1}{\subsection*{Node v2-1.3 Defendant nurses on Ward 2}}
 The defendant nurses were named, and provided witness statements.
\vskip .5ex\vbox{\small \hskip 2em\\$\rightarrow$ \hyperlink{6}{\colorflag{white}~v2-3.5 Ward PCs}\hskip 2em\\$\rightarrow$ \hyperlink{4}{\colorflag{white}~v2-2.4 XceedPros on Ward 2}\hskip 2em\\$\rightarrow$ \hyperlink{3}{\colorflag{blue}~v2-8.4 Nurse-written paper records (never disclosed)}}

\hypertarget{Abbott_engineer}{\subsection*{ \colorflag{blue} Node v3-2.2 Unsupervised Abbott engineer}}
\textcolor{blue}{\sf\textbf{Keyphrases:} Corruption of evidence; PrecisionWeb database.}
\vskip 1ex
 \warn{Mr\@. Nick Reece, a PrecisionWeb support Specialist employed by Abbott Diabetes Care, had been asked by the hospital to help when the Princess of Wales PrecisionWeb database faced problems in 2013. Mr.\ Reece then worked unsupervised on the PrecisionWeb database, apparently taking no notes and certainly not following any forensic process.}

Reece gave evidence that he had failed to take notes on exactly what he did when editing, deleting, and modifying data in the PrecisionWeb database. He admitted he had deleted critical data, which would have created the impression that nurses had been negligent not performing blood glucose tests.

\warn{This deletion of computer evidence explains the discrepancy in patient records that the prosecution claims had been wholly and entirely caused by the nurses' criminal negligence.}
\vskip .5ex\vbox{\small \hskip 2em\\$\rightarrow$ \hyperlink{unknownEdits}{\colorflag{blue}~v2-3.3 Unknown manual edits to PrecisionWeb database}}

\hypertarget{networks}{\subsection*{ \colorflag{white} Node v2-2.3 Ward networks\\\hskip 2em --- (Group: Hospital IT Systems) }}
\textcolor{blue}{\sf\textbf{Keyphrase:} Error logs.}
\vskip 1ex
 Ward networks may have had outages during the relevant period. The prosecution should supply network logs to confirm relevant ward networks had no outages over the relevant period.
\vskip .5ex\vbox{\small \hskip 2em\\$\rightarrow$ \hyperlink{5}{\colorflag{white}~v2-3.4 XceedPro dock on ward}}

\hypertarget{4}{\subsection*{ \colorflag{white} Node v2-2.4 XceedPros on Ward 2\\\hskip 2em --- (Group: Hospital IT Systems) }}
 The nurse takes an XceedPro to a patient, and they must then tap the XceedPro with their ID card, then scan the patient's ID card. Then the patient is pricked to obtain a small blood sample, which is inserted into the XceedPro. The XceedPro records the IDs, the date and time, and the blood glucose level. Nurses may share ID cards, or they may ``double tap'' and use their ID card twice instead of using the patient's ID card.

See node \hyperlink{17}{\raise 1pt \hbox{\fbox{\sf\tiny FROM}} v2-2.1 Abbott
labs}: \warn{Manufacturer evidence says XceedPros have numerous bugs; however none of the documented bugs lose patient data.}
\vskip .5ex\vbox{\small \hskip 2em\\$\rightarrow$ \hyperlink{5}{\colorflag{white}~v2-3.4 XceedPro dock on ward}}

\hypertarget{6unknown}{\subsection*{ \colorflag{yellow} Node v2-2.5 No evidence explaining scope of Ward PCs\\\hskip 2em --- (Group: Hospital IT Systems) }}
\textcolor{blue}{\sf\textbf{Keyphrase:} PrecisionWeb database.}
\vskip 1ex
\warn{The Ward PCs can be used by nurses to review the PrecisionWeb database.
Beyond that there is no evidence to explain what else these PCs can do.}

Can they be used to edit or delete existing records on PrecisionWeb? We don't know.
\vskip .5ex\vbox{\small \hskip 2em\\$\rightarrow$ \hyperlink{6}{\colorflag{white}~v2-3.5 Ward PCs}}

\hypertarget{manual}{\subsection*{Node v2-3.2 PrecisionWeb operators manual}}
\textcolor{blue}{\sf\textbf{Keyphrase:} PrecisionWeb database.}
\vskip 1ex
 \warn{The operators manual for PrecisionWeb says it \textbf{should not be used for clinical purposes}. All the evidence for this case assumed the clinical records in PrecisionWeb were reliable, despite the manufacturer's warning.}
\vskip .5ex\vbox{\small \hskip 2em\\$\rightarrow$ \hyperlink{7}{\colorflag{blue}~v2-5.2 Abbott PrecisionWeb database}}

\hypertarget{unknownEdits}{\subsection*{ \colorflag{blue} Node v2-3.3 Unknown manual edits to PrecisionWeb database\\\hskip 2em --- (Group: Hospital IT Systems) }}
\textcolor{blue}{\sf\textbf{Keyphrase:} PrecisionWeb database.}
\vskip 1ex
 Written evidence shows that there have been unmanaged manual edits and manual maintenance of the PrecisionWeb database.

The impact of these edits on the case is as yet unknown.

\warn{If the technicians or engineers (potentially including Abbott maintenance staff) responsible for managing PrecisionWeb do not provide further more specific evidence, we will have no idea what they did to the database, potentially corrupting or deleting critical data it records that is likely to be relevant to the case.}
\vskip .5ex\vbox{\small \hskip 2em\\$\rightarrow$ \hyperlink{7}{\colorflag{blue}~v2-5.2 Abbott PrecisionWeb database}}

\hypertarget{5}{\subsection*{ \colorflag{white} Node v2-3.4 XceedPro dock on ward\\\hskip 2em --- (Group: Hospital IT Systems) }}
\textcolor{blue}{\sf\textbf{Keyphrase:} Testing.}
\vskip 1ex
 \warn{No evidence of the reliability of the XceedPro dock (for example, hospital test results, or manufacturer claims of reliability, or even daily checks of integrity) has been offered. Its reliable use depends on hospital network reliability, and no evidence of the reliability of the hospital network has been offered.} 
\vskip .5ex\vbox{\small \hskip 2em\\$\rightarrow$ \hyperlink{7}{\colorflag{blue}~v2-5.2 Abbott PrecisionWeb database}}

\hypertarget{6}{\subsection*{ \colorflag{white} Node v2-3.5 Ward PCs\\\hskip 2em --- (Group: Hospital IT Systems) }}
 The ward has PCs where nurses can review patient information.

\warn{It is not clear whether nurses can edit data, for instance in case two patients were mixed up with the XceedPros (e.g., by putting the wrong test strip in or using one with a second patient) --- and if so, whether edits must be done by the nurse (or person with the same ID) who originally took the readings.}

\warn{We do not know who had access to the Ward PCs (whether nurses or hospital operators).}
\vskip .5ex\vbox{\small \hskip 2em\\$\rightarrow$ \hyperlink{7}{\colorflag{blue}~v2-5.2 Abbott PrecisionWeb database}\hskip 2em\\$\rightarrow$ \hyperlink{8}{\colorflag{blue}~v2-6.2 Lots of unknown complex middleware}}

\hypertarget{13}{\subsection*{ \colorflag{white} Node v2-5.1 Hospital computer operators}}
 \warn{We have no idea what the hospital computer system operators did, how they were managed, or what their security protocols (if any) were.}

Note how (using Statechart conventions for diagrams) a \emph{single\/} arrow from this node (\hyperlink{13}{\raise 1pt \hbox{\fbox{\sf\tiny HERE}} v2-5.1 Hospital
computer
operators}) to the group \hyperlink{hospital}{\raise 1pt \hbox{\fbox{\sf\tiny TO}} v2-9.1 Hospital IT Systems} is sufficient to show that the computer operators affect \emph{all\/} nodes inside the box.\footnote{Statecharts, invented by David Harel, are well-known diagrams widely used in computer science.}
\vskip .5ex\vbox{\small \hskip 2em\\$\rightarrow$ \hyperlink{hospital}{\hphantom{\colorflag{white}}~v2-9.1 Hospital IT Systems}\hskip 2em\\$\rightarrow$ \hyperlink{14}{\colorflag{yellow}~v2-7.1 No evidence of any testing ever performed}}

\hypertarget{7}{\subsection*{ \colorflag{blue} Node v2-5.2 Abbott PrecisionWeb database\\\hskip 2em --- (Group: Hospital IT Systems) }}
\textcolor{blue}{\sf\textbf{Keyphrase:} PrecisionWeb database.}
\vskip 1ex
 PrecisionWeb is a proprietary database designed by Abbott Diabetes Care Inc.\ to monitor the XceedPros, such as the state of their battery health. The PrecisionWeb database also records date and time, and nurse IDs, patient IDs, blood glucose levels, etc.

The PrecisionWeb manual warns that it is not designed for clinical use. Nevertheless the Princess of Wales hospital used PrecisionWeb for clinical purposes. At the Princess of Wales the blood glucose readings (amongst other data) are copied, via middleware (node \hyperlink{8}{\raise 1pt \hbox{\fbox{\sf\tiny TO}} v2-6.2 Lots of unknown
complex middleware}) to the main hospital databases.


\warn{If the manufacturers say it is not reliable enough for clinical use, why is PrecisionWeb assumed good enough for evidence?}


\warn{The prosecutors must therefore establish that PrecisionWeb \emph{and its sources of data\/} are maintained and used to an appropriate standard to ensure it provides reliable evidence. They should also state what they do to overcome the manufacturer's explicit warnings about its quality.}

\vskip .5ex\vbox{\small \hskip 2em\\$\rightarrow$ \hyperlink{backups}{\colorflag{blue}~v2-8.2 No known backups that could have provided evidence}\hskip 2em\\$\rightarrow$ \hyperlink{8}{\colorflag{blue}~v2-6.2 Lots of unknown complex middleware}\hskip 2em\\$\rightarrow$ \hyperlink{11}{\colorflag{blue}~v2-6.3 Police forensic database system}\hskip 2em\\$\rightarrow$ \hyperlink{10}{\colorflag{blue}~v2-6.1 Frequent crashes documented}}

\hypertarget{forensicProblems}{\subsection*{ \colorflag{yellow} Node v2-5.3 Written evidence shows forensic methods used were unreliable}}
\textcolor{blue}{\sf\textbf{Keyphrases:} Backups; CSV; Cybersecurity; PrecisionWeb database.}
\vskip 1ex
 \warn{The police visited the hospital and exported CSV data from the hospital's PrecisionWeb database to an unencrypted USB stick, a process that is insecure.}

Although there were opportunities for the police to collect incorrect or incomplete evidence, or to corrupt the evidence that they had collected, there is no nothing to suggest that the police checked their final computer evidence against the original hospital sources. The police process took no precautions to detect or mitigate the effect of possible cyberattack (or other database problems, such as hardware faults, system crashes --- which were known to occur --- or operator error), such as comparing their collected data with backups.

Although the police used Excel to manage the evidence (which can delete data, columns and rows, etc, without leaving any record of changes) there is no record to show that they checked their final evidence against the original data.

\warn{The final evidence was presented to the court on both encrypted CDs and USB sticks \emph{and was explicitly claimed to be forensic quality evidence}.}
\vskip .5ex\vbox{\small \hskip 2em\\$\rightarrow$ \hyperlink{11}{\colorflag{blue}~v2-6.3 Police forensic database system}}

\hypertarget{10}{\subsection*{ \colorflag{blue} Node v2-6.1 Frequent crashes documented\\\hskip 2em --- (Group: Hospital IT Systems) }}
\textcolor{blue}{\sf\textbf{Keyphrase:} PrecisionWeb database.}
\vskip 1ex
 \warn{Evidence revealed in court was that the PrecisionWeb database (or the server it ran on) frequently crashed. The evidence said nothing about the impact of crashes on the integrity of the database; presumably at least the transactions pending when crashes happened would be lost.}


\hypertarget{8}{\subsection*{ \colorflag{blue} Node v2-6.2 Lots of unknown complex middleware\\\hskip 2em --- (Group: Hospital IT Systems) }}
\textcolor{blue}{\sf\textbf{Keyphrases:} Error logs; PrecisionWeb database; Testing.}
\vskip 1ex
 PrecisionWeb (node \hyperlink{7}{\raise 1pt \hbox{\fbox{\sf\tiny FROM}} v2-5.2 Abbott
PrecisionWeb
database}) is not interoperable with other hospital systems, so middleware Conworx is used to interface PrecisionWeb to the main patient databases.

\warn{No evidence of the reliability of Conworx and other middleware (for example, hospital test results, or manufacturer claims of reliability) has been mentioned in evidence.}

\warn{The hospital refused access to technicians, which meant we have no idea of or the extent of the middleware or how it is supposed to work.}

\warn{Such complex multi-manufacturer systems will have bugs, so it is very surprising not to have error logs (etc) presented in routine statements.}
\vskip .5ex\vbox{\small \hskip 2em\\$\rightarrow$ \hyperlink{unknowndatabaselogs}{\colorflag{blue}~v2-8.3 No evidence on any system reliability}\hskip 2em\\$\rightarrow$ \hyperlink{9}{\colorflag{blue}~v2-7.2 Main hospital databases}}

\hypertarget{11}{\subsection*{ \colorflag{blue} Node v2-6.3 Police forensic database system}}
\textcolor{blue}{\sf\textbf{Keyphrases:} CSV; PrecisionWeb database.}
\vskip 1ex
 \warn{The police used PrecisionWeb unsupervised, exported data to CSV format.}\footnote{Hospital policy has since been revised.}

The police certainly examined the PrecisionWeb data in Excel, as their written evidence says their first attempts crashed Excel (presumably because of the size of the Excel spreadsheet). I do not understand why the police did not use SQL (or any other decent database), as they could simply have copied the SQL PrecisionWeb database rather than relying on PrecisionWeb to convert it to CSV\@.

\warn{CSV is a \emph{very\/} unsuitable data format to choose for evidence, as there is no way to detect edited, deleted, or inserted data.}

Whatever CSV data they selected, they copied to a USB stick. This stick was then taken by car to a police forensic environment, and copied to an encrypted secure disk drive. Subsequently, unencrypted CDs were burned from the files on the secure disc.

The police evidence claimed their process was forensic --- but it was only forensic \emph{after\/} copying CSV data to the secure disk drive, and provided that no police staff with password access modified the data.
\vskip .5ex\vbox{\small \hskip 2em\\$\rightarrow$ \hyperlink{12}{\colorflag{blue}~v2-7.3 Main police evidence}}

\hypertarget{14}{\subsection*{ \colorflag{yellow} Node v2-7.1 No evidence of any testing ever performed}}
\textcolor{blue}{\sf\textbf{Keyphrases:} PrecisionWeb database; Testing.}
\vskip 1ex
 \warn{The hospital presented no evidence that it knew that any of its systems were reliable. However, the hospital did present evidence that PrecisionWeb regularly crashes (node \hyperlink{10}{\raise 1pt \hbox{\fbox{\sf\tiny FROM}} v2-6.1 Frequent crashes
documented}).}
\vskip .5ex\vbox{\small \hskip 2em\\$\rightarrow$ \hyperlink{15}{\colorflag{yellow}~v2-8.1 No evidence of error logs or audits}}

\hypertarget{9}{\subsection*{ \colorflag{blue} Node v2-7.2 Main hospital databases\\\hskip 2em --- (Group: Hospital IT Systems) }}
\textcolor{blue}{\sf\textbf{Keyphrases:} Error logs; PrecisionWeb database.}
\vskip 1ex
 The hospital databases are managed by hospital technicians. None provided evidence statements, and none were called to give evidence.

\warn{Such complex systems will have bugs, so it is very surprising not to have error logs (etc) presented in routine statements.}

\warn{It is curious that the discrepancies were originally discovered from the main hospital databases, not from PrecisionWeb. The police never took evidence from the main hospital databases.}
\vskip .5ex\vbox{\small \hskip 2em\\$\rightarrow$ \hyperlink{unknowndatabaselogs}{\colorflag{blue}~v2-8.3 No evidence on any system reliability}\hskip 2em\\$\rightarrow$ \hyperlink{backups}{\colorflag{blue}~v2-8.2 No known backups that could have provided evidence}\hskip 2em\\$\rightarrow$ \hyperlink{3}{\colorflag{blue}~v2-8.4 Nurse-written paper records (never disclosed)}}

\hypertarget{12}{\subsection*{ \colorflag{blue} Node v2-7.3 Main police evidence}}
 \warn{A joint prosecution/defence expert witness report (``Experts' Joint Statement on Matters Agreed and/or Disagreed in regard to Part 35 12 (3) CPR'') lists many problems, including that the CDs CH/15 the police provided to the prosecution and defence experts were different, raising serious questions of police forensic process
\cite{POW-resources}.}

Refer to node \hyperlink{11}{\raise 1pt \hbox{\fbox{\sf\tiny FROM}} v2-6.3 Police forensic
database system}. 


\hypertarget{15}{\subsection*{ \colorflag{yellow} Node v2-8.1 No evidence of error logs or audits}}
\textcolor{blue}{\sf\textbf{Keyphrases:} Audit; Error logs.}
\vskip 1ex
 \warn{It is astonishing there is no evidence that the hospital ever checked whether their computer systems were functioning normally.}

\warn{It is also astonishing the police recorded data from hospital systems without confirming the reliability of the data, and did so without hospital supervision.}


\hypertarget{backups}{\subsection*{ \colorflag{blue} Node v2-8.2 No known backups that could have provided evidence\\\hskip 2em --- (Group: Hospital IT Systems) }}
\textcolor{blue}{\sf\textbf{Keyphrases:} Backups; PrecisionWeb database.}
\vskip 1ex
 If backups had been taken (we do not know) then it would have been natural to check if the alleged missing nurse data had ever been stored on the main PrecisionWeb database.

Backups would have shown whether failures happened in PrecisionWeb, or earlier for instance over arrows A (as the prosecution argued), B, C or D.


\hypertarget{unknowndatabaselogs}{\subsection*{ \colorflag{blue} Node v2-8.3 No evidence on any system reliability\\\hskip 2em --- (Group: Hospital IT Systems) }}
The hospital should have audited or kept logs on PrecisionWeb. There is no evidence (yet) that it did.


\hypertarget{3}{\subsection*{ \colorflag{blue} Node v2-8.4 Nurse-written paper records (never disclosed)}}
 Contemporaneously with treating patients, nurses should review the XceedPro they have used and copy its data to paper notes. The police never provided the nurse-written records.

\warn{The XceedPro can record up to 2,500 patient records and then it will start overwriting records. As I never had access to the relevant XceedPros, I do not know whether this bug was relevant: had an XceedPro recorded about 2,500 records it \emph{could\/} have given the impression the nurse had not performed tests the nurse had documented on paper notes.}


\hypertarget{hospital}{\subsection*{Node v2-9.1 Hospital IT Systems}}
\textcolor{blue}{\sf\textbf{Keyphrase:} Error logs.}
\vskip 1ex
 \warn{We know nothing about the management of the hospital systems and any impact on the evidence.} \warn{In a hospital (or any other critical environment) one would expect at least daily checks of reliable performance or error logs. The REED diagram shows there were no such checks.}
\vskip .5ex\vbox{\small \hskip 2em\\$\rightarrow$ \hyperlink{14}{\colorflag{yellow}~v2-7.1 No evidence of any testing ever performed}}\hypertarget{component2-narrative}{\section*{Node narrative evidence for component 2}}
\begin{description}
\item[$\rightarrow$] \hyperlink{component1-narrative}{All narrative for component 1}
\end{description}


\hypertarget{16}{\subsection*{ \colorflag{red} Node v2-1.1 Wrong XceedPros seized by police\\\hskip 2em --- (Group: Police handling of XceedPro evidence) }}
\textcolor{blue}{\sf\textbf{Keyphrase:} PrecisionWeb database.}
\vskip 1ex
 The PrecisionWeb data records XceedPro serial numbers, and shows that XceedPros routinely move around the hospital.

\warn{The PrecisionWeb data confirms that the police seized all three XceedPros that happened to be on Ward 2 on the date when they visited it. However, the seizure did not include any XceedPros that had been used by the defendants at the relevant times.}

\warn{In addition to mistakenly seizing the wrong XceedPros, the police sent them to Abbott (node \hyperlink{17}{\raise 1pt \hbox{\fbox{\sf\tiny TO}} v2-2.1 Abbott
labs}) to be checked for reliability. Abbott unsurprisingly confirmed the XceedPros worked correctly --- it would have been better to chose an independent organization to check the XceedPros.}

\warn{Furthermore, Abbott's evidence says nothing about the reliability of the XceedPros \emph{actually\/} used by the nurses. Checking the wrong XceedPros is irrelevant to the Prosecution's contention that the nurses fabricated meter readings.}

Reference \cite{POW} explains how the police came to seize the wrong XceedPros.

\vskip .5ex\vbox{\small \hskip 2em\\$\rightarrow$ \hyperlink{17}{\hphantom{\colorflag{white}}~v2-2.1 Abbott labs}\hskip 2em\\$\rightarrow$ \hyperlink{18}{\colorflag{red}~v2-3.1 Police summary of wrong XceedPro evidence}}

\hypertarget{17}{\subsection*{Node v2-2.1 Abbott labs\\\hskip 2em --- (Group: Police handling of XceedPro evidence) }}
\textcolor{blue}{\sf\textbf{Keyphrase:} PrecisionWeb database.}
\vskip 1ex
 The US company Abbott manufactured the XceedPros and the PrecisionWeb database, both critical components of this case.

The police sent the 3 seized XceedPros to Abbott to determine if they worked correctly. It is surprising the police chose Abbott themselves to check the XceedPros, as Abbott have a vested interest in saying their own products are reliable. Moreover, the police sent the wrong XceedPros to Abbott, as they had seized the glucometers on the ward when they visited, not the glucometers that the nurses had originally used.

The Abbott labs found no relevant problems with the XceedPros, but they did confirm that all XceedPros had bugs.

However, none of the bugs Abbott described would affect or cause loss of data, unless the XceedPros had been used to try to record over 2,500 tests, which is considerably higher than any XceedPro at the Princess of Wales had ever recorded (at least as established by the police's PrecisionWeb evidence).
\vskip .5ex\vbox{\small \hskip 2em\\$\rightarrow$ \hyperlink{18}{\colorflag{red}~v2-3.1 Police summary of wrong XceedPro evidence}}

\hypertarget{18}{\subsection*{ \colorflag{red} Node v2-3.1 Police summary of wrong XceedPro evidence\\\hskip 2em --- (Group: Police handling of XceedPro evidence) }}
 \warn{Because of the problems noted under node \hyperlink{16}{\raise 1pt \hbox{\fbox{\sf\tiny FROM}} v2-1.1 Wrong XceedPros
seized by police}, this evidence was excluded.}


\hypertarget{xceedpro}{\subsection*{Node v2-4.1 Police handling of XceedPro evidence}}
The police examined the wrong XceedPros. The evidence presented on XceedPros is not relevant to the case.
\section*{Arrow narrative evidence}
\subsection*{Arrow ?\\
\setbox0=\hbox{$\rightarrow$~}\hbox to \wd0{}v2-6.2 \hyperlink{8}{Lots of unknown complex middleware}\\\copy0{}v2-5.2 \hyperlink{7}{Abbott PrecisionWeb database}}
\textcolor{blue}{\sf\textbf{Keyphrase:} Testing.}
\vskip 1ex
Arrows marked ``\textbf{?}'' indicate evidence flows that we \emph{should\/} know about, but no relevant information has been made available, despite requests. We think the evidence probably does not exist because (we surmise that) the IT operators probably did not do the appropriate work, such as testing (see \hyperlink{14}{\raise 1pt \hbox{\fbox{\sf\tiny TO}} v2-7.1 No evidence
of any testing
ever performed}).
\subsection*{Arrow  A\\
\setbox0=\hbox{$\rightarrow$~}\hbox to \wd0{}v2-1.3 \hyperlink{1}{Defendant nurses on Ward 2}\\\copy0{}v2-2.4 \hyperlink{4}{XceedPros on Ward 2}}
A nurse uses an XceedPro to take a sample from a specific patient, and the XceedPro records the nurse's and the patient's IDs.
\subsection*{Arrow  B\\
\setbox0=\hbox{$\rightarrow$~}\hbox to \wd0{}v2-1.2 \hyperlink{2}{Patients}\\\copy0{}v2-2.4 \hyperlink{4}{XceedPros on Ward 2}}
A blood sample from a patient enters an XceedPro.
\subsection*{Arrow  C\\
\setbox0=\hbox{$\rightarrow$~}\hbox to \wd0{}v2-2.4 \hyperlink{4}{XceedPros on Ward 2}\\\copy0{}v2-3.4 \hyperlink{5}{XceedPro dock on ward}}
The XceedPros provide date, time, XceedPro ID, nurse ID, patient ID, blood glucose level, etc.
\subsection*{Arrow  D\\
\setbox0=\hbox{$\rightarrow$~}\hbox to \wd0{}v2-3.4 \hyperlink{5}{XceedPro dock on ward}\\\copy0{}v2-5.2 \hyperlink{7}{Abbott PrecisionWeb database}}
Should be identical to the XceedPro data from \hyperlink{4}{\raise 1pt \hbox{\fbox{\sf\tiny FROM}} v2-2.4 XceedPros
on Ward 2} via arrow C plus a timestamp of upload.
\subsection*{Arrow  E\\
\setbox0=\hbox{$\rightarrow$~}\hbox to \wd0{}v2-5.2 \hyperlink{7}{Abbott PrecisionWeb database}\\\copy0{}v2-6.3 \hyperlink{11}{Police forensic database system}}
\textcolor{blue}{\sf\textbf{Keyphrases:} CSV; PrecisionWeb database.}
\vskip 1ex
The police manually exported a CSV file from the PrecisionWeb database (supposedly) of every record derived from all XceedPros in the hospital over the relevant period of time. This was then copied to a USB stick and taken by car to be copied to the police system, from where multiple CDs were made for copying to the prosecution and defence. \warn{CSV is a non-forensic and very unreliable data format, and the police made no use of checksums or other techniques to assure that the final CDs correctly held the data from PrecisionWeb.}
\subsection*{Arrow   F\\
\setbox0=\hbox{$\rightarrow$~}\hbox to \wd0{}v2-6.3 \hyperlink{11}{Police forensic database system}\\\copy0{}v2-7.3 \hyperlink{12}{Main police evidence}}
Should be a collection of all data starting at \hyperlink{4}{\raise 1pt \hbox{\fbox{\sf\tiny FROM}} v2-2.4 XceedPros
on Ward 2} (it wasn't; see figure 7 in the REED paper for the reasons) but crucially the police evidence was incomplete since it provided no way to check consistency with the data from \hyperlink{4}{\raise 1pt \hbox{\fbox{\sf\tiny FROM}} v2-2.4 XceedPros
on Ward 2}.




\end{document}
