Use reed (no parameters) to summarise flags

Syntax summary
==============

# comment to end of line
# Hash "#" is used in REED files, and is also used for comments in this syntax summary

### syntax notation...

x ::= y # define non-terminal x to be y
x | y   # x or y
{ x }   # grouping in syntax definitions
x+      # at least one x. Short for: x | xx | xxx...
x++     # at least one x. Short for: x | x x | x x x... separated by spaces
[ x ]   # optional x
        # spaces/blanks can be put between all syntactic items and used freely for layout 

### REED syntax...

id ::= { letter | digit | underline }+ | Cstring | Herestring
Cstring ::= "C string, \\n for newline, \\\\ for backslash etc" 
Herestring ::= <<< arbitrary-terminator\n                           characters (no escapes needed) over any number of lines\n                       same arbitrary-terminator at start of line

### narrative strings in notes are translated to HTML and Latex, and can include:

<html> ...  following text is copied to HTML, but skipped in Latex
<latex> ... following text is copied to Latex, but skipped in HTML
<both> ...  following text is mixed basic HTML and Latex (use flag -rules to see translation rules)
[[[id]]]!replace with id's name and make a cross-reference link (in both HTML and Latex)

# reserved words can be used as ids if written as strings
# otherwise, id xyz and string "xyz" are equivalent
# so red is reserved, but "red" can be used arbitrarily
# string notations allow more characters to be used

### the following are definitions of non-terminals

arrow ::= -> | <- | <->
color ::= black | blue | green | red | white | yellow
id-or-star ::= id | *
node-or-arrow ::= id | id { arrow id }+
node-or-arrow-list ::= node-or-arrow | ( node-or-arrow++ )
document-details ::= abstract | author | date | title

### the following are REED statements
### all terminals below are reserved words

check id1 => id2!Check there is a path from id1 to id2
direction { LR | RL | TB | BT }!Primary direction of graph is left-right, right-left, top-bottom, or bottom-top
document-details id!Specify document details. author id can be used multiple times for multiple authors
group node-or-arrow-list is id!The nodes are grouped, and the group is named
highlight color [ cascade ] is id!Explain color as id, optionally cascading when it is used
highlight id is color!Highlight node id with color
htmldefinitions id!Insert id at start of <body> in HTML file; multiple htmldefinitions append it
latexdefinitions id!Insert id at start of Latex file; multiple latexdefinitions append id 
new style id1 is id2!Define id1 as a style, using Dot syntax for id2!(e.g., new style blue18 is "fontcolor=blue; fontsize=18")
node-or-arrow [ is id ]!Create nodes or arrows, with optional display name id
note node-or-arrow-list [ is id1 [;] ] [ author id2 [;] ] id3!Provide narrative note id3 for the nodes or arrows, optionally naming them id1
numbering ( { ( id-or-star+ ) }++ )!Make node references row.column numbers following the grid layout
ref id1 is id2!Set the reference of node id1 to be id2
rows [ TOC ] ( { id ( id+ ) }++ )!Draw a graph with a structured layout, with optional TOC (Theory of Change) styling
style node-or-arrow-list is id!Set style of nodes or arrows, either to a named style or to an explicit Dot specification
tags id1 id2!Only process REED code between id1 (begin) and id2 (end)
version id!Subsequent REED code is version id (see flag v=value)

;!semicolons are empty statements, and can be placed freely between statements!... and in note statements for legibility
override ...!Permit conflicting commands (e.g., title) to override earlier commands (used with version id and flag v=id)

### Apart from override, tags, and version the order of commands does not matter
### Except author, htmldefinitions, and latexdefinitions append in REED file order

